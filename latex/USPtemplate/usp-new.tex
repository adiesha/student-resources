\documentclass[11pt]{article}
\usepackage{uspstyle}

\begin{document}

\maketitle

% The content of your research proposal, including text, figures and images, 
% should not exceed 5 pages, typed, double-spaced (additional pages may be 
% included for your works-cited/references).

\section{Introduction}
% Provide a statement of the objective(s) and the anticipated significance of the 
% work to your field of study. What problems will be investigated? What hypothesis 
% will be tested? We suggest that the introduction begin with a brief description 
% of the project in general terms before the more technical aspects of the project 
% are discussed. 
\todo{put some sort of high-level introduction here.}
here is how you cite a figure (be sure to updated the ref here ... and the 
label inside the figure
%Figure~\ref{fig:TODO-name}

TO cite something, use~\cite{fasy2014statistical}.

% \begin{figure}
%  \centering
%  \includegraphics[height=1.25in]{TODO:relative-path}
%  \caption{TODO.}\label{fig:TODO-name}
% \end{figure}

\section{Background}
% Provide a brief review of the work that has been done in the project 
% area together with complete references in appropriate professional style. This 
% section should also include any personal information about you that would 
% indicate to the reviewers your qualifications for successfully completing this 
% project, including a statement of how the project will contribute to your 
% academic and career goals.
\todo{discuss related work, and why you are qualified to work on this.  Perhaps 
make subsections.}

\section{Methods}
% Provide a detailed description of the research methods that you will 
% use in the project. This should include a justification for the specific 
% approach that you will use. For example, how do the methods answer the questions 
% that have been posed, test the hypothesis, or lead to the desired goal?
% Timeline: Provide dates for the initiation and completion of each phase of the 
% project. Attempt to lay out a reasonable schedule taking into consideration all 
% phases of the research and final deliverables.

\section{Collaboration with Faculty Sponsor}
% Provide a description of the way you and 
% your faculty sponsor will collaborate on the project. The faculty sponsor should 
% play a significant role in responding to your ideas, providing advice for new 
% directions and resources, discussing the implications of the results, and 
% helping you prepare for your public presentation. Will there be regularly 
% scheduled meetings between you and your sponsor? Explain how the project relates 
% to the ongoing work of your sponsor, if this is the case.


\todo{Does the following apply?  Report on Previous Research Experience (please 
save and upload this as a 
separate document): If you have done any previous research as an undergraduate 
you must include a 1-2 page (double-spaced) summary of your research results or 
creative products.Please note-if you have received funding from USP or INBRE 
your proposal will not be considered unless you complete this section.}

% Please draft your proposal in a format that is appropriate for your academic 
% discipline (i.e. MLA, APA, etc.) - consult your mentor if you have questions 
% about what format is most appropriate to your field of study
%%%%%%%%%%%%%%%%%%%%%%%%%%%%%%%%%%%%%%%%%%%%
%% BIBLIOGRAPHY
 \newpage
 \bibliographystyle{acm}
 \bibliography{references}
%%%%%%%%%%%%%%%%%%%%%%%%%%%%%%%%%%%%%%%%%%%%
% References Cited (include in an additional page within the project proposal): 
% Include a list of any literature that you have cited in the proposal. Nearly 
% all 
% good science and engineering proposals cite papers reporting related results, 
% describing the methods to be used or providing background information. Please 
% note-the review panel rarely recommends funding for proposals without adequate
% references.



\end{document}


